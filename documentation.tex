\documentclass{article}
\usepackage{graphicx}
\usepackage{hyperref}
\usepackage{listings}
\usepackage{xcolor}
\usepackage{geometry}
\usepackage{booktabs}

\geometry{a4paper, margin=1in}
\definecolor{codegreen}{rgb}{0,0.6,0}
\definecolor{codegray}{rgb}{0.5,0.5,0.5}
\definecolor{codepurple}{rgb}{0.58,0,0.82}

\lstset{
    basicstyle=\ttfamily\footnotesize,
    breaklines=true,
    frame=single,
    numbers=left,
    numbersep=5pt,
    showstringspaces=false,
    keywordstyle=\color{blue},
    commentstyle=\color{codegreen},
    stringstyle=\color{codepurple}
    
\title{Pulse Class Documentation}
\author{Quantum Simulation Toolkit}
\date{\today}

\begin{document}
\maketitle

\section{Overview}
The \texttt{Pulse} class models time-dependent laser parameters for quantum simulations:
\begin{itemize}
    \item \(\Omega(t)\): Rabi frequency (MHz)
    \item \(\phi(t)\): Phase (radians)
    \item \(\delta(t)\): Detuning (MHz)
\end{itemize}

\section{Class Features}
\begin{itemize}
    \item Independent configuration of \(\Omega\), \(\phi\), and \(\delta\)
    \item Pre-built waveform shapes: \texttt{constant}, \texttt{gaussian}, \texttt{linear}
    \item Custom mathematical functions support
    \item Integrated visualization
\end{itemize}

\section{Initialization}
\begin{lstlisting}[language=Python]
pulse = Pulse()  # No required arguments
\end{lstlisting}

\section{Configuration Methods}
Configure parameters independently using:
\begin{lstlisting}[language=Python]
pulse.set_omega(shape: str, **kwargs)
pulse.set_phi(shape: str, **kwargs)
pulse.set_delta(shape: str, **kwargs)
\end{lstlisting}

\subsection{Supported Waveform Types}
\begin{tabular}{@{}lll@{}}
    \toprule
    \textbf{Shape} & \textbf{Parameters} & \textbf{Description} \\
    \midrule
    \texttt{constant} & \texttt{value: float} & Fixed value \\
    \texttt{gaussian} & \texttt{amp, t0, sigma: float} & Gaussian pulse \\
    \texttt{linear} & \texttt{slope, intercept: float} & Linear ramp \\
    \texttt{custom} & \texttt{func: Callable} & User-defined function \\
    \bottomrule
\end{tabular}

\section{Visualization}
Plot parameters over time:
\begin{lstlisting}[language=Python]
pulse.plot(t_range=(0, 5), num_points=200)
\end{lstlisting}

\begin{figure}[h]
    \centering
    \includegraphics[width=0.8\textwidth]{pulse_plot.png}
    \caption{Sample pulse visualization showing Gaussian \(\Omega(t)\), linear \(\phi(t)\), and constant \(\delta(t)\)}
\end{figure}

\section{Example Usage}
\subsection{Basic Configuration}
\begin{lstlisting}[language=Python]
pulse = Pulse()
pulse.set_omega('gaussian', amp=10.0, t0=2.5, sigma=0.5)
pulse.set_phi('linear', slope=np.pi/2, intercept=0.0)
pulse.set_delta('constant', value=1.0)
\end{lstlisting}

\subsection{Custom Waveform}
\begin{lstlisting}[language=Python]
# Custom delta(t) = sin(2π*0.5*t)
pulse.set_delta('custom', 
               func=lambda t, p: np.sin(2*np.pi*0.5*t))
\end{lstlisting}

\section*{Notes}
\begin{itemize}
    \item Time units: microseconds (\(\mu\)s)
    \item Frequency units: MHz for \(\Omega\) and \(\delta\)
    \item Phase values not automatically wrapped to \(2\pi\)
    \item Custom functions must accept \texttt{(t, params)} arguments
\end{itemize}

\end{document}